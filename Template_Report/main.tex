% Esta plantilla ha sido diseñada por Daniel del Río Velilla, profesor en la Escuela de Aeronáutica y del Espacio, UPM.

\documentclass[11pt,a4paper,titlepage]{article}

%   ---   DEFINICION DEL TRABAJO   ---   %

\newcommand{\Project}{Plantillas de \LaTeX}
\newcommand{\ProjectTitle}{Plantilla de \LaTeX\, para informes}
\newcommand{\ProjectSubject}{Grado de Ingeniería Aeroespacial}
\newcommand{\ProjectAutor}{ Alumno 1 \\
                            & Alumno 2}
\newcommand{\ProjectTutor}{ Tutor 1 \\
                            & Tutor 2}
\newcommand{\ProjectLocation}{Madrid}
\newcommand{\ProjectDate}{19 de septiembre de 2022}
% Descomentar si se tiene un repositorio del trabajo
% \newcommand{\ProjectGitHub}{url}



%   ---   INCLUIR ARCHIVOS DE CONFIGURACION   ---   %

%----------------------------------------------------------------------------------------
%	User PACKAGES
%----------------------------------------------------------------------------------------

\usepackage{pdfpages}

\usepackage{listings}
\usepackage{ctable}
\usepackage{xcolor}
\usepackage{xcolor,colortbl}


\usepackage{algorithm}
\usepackage{algorithmic}

\usepackage{mdframed}

\usepackage{afterpage}

\newacronym{mc}{MC}{Materiales Compuestos}
\newacronym{cfrp}{CFRP}{Carbon Fiber Reinforced Polymer}
\newacronym{udpp}{UDPP}{Unidirectional Prepeg}
\newacronym{fr}{FR}{Filament Ripper}
\newacronym{rrf}{RRF}{RepRap Firmware}
\newacronym{dwc}{DWC}{Duet Web Control}
\newacronym{sbc}{SBC}{Single Board Computer}
\newacronym{rsp}{RSP4}{Raspberry Pi 4}
\newacronym{cad}{CAD}{Computer Aided Desing}
\newacronym{rpas}{RPAS}{Remotely Piloted Aircraft System}
\newacronym{ba}{BA}{Borde de ataque}
\newacronym{bs}{BS}{Borde de salida}
\newacronym{eop}{EoP}{Edge of Part}
\newacronym{eeop}{EEoP}{Engineering Edge of Part}
\newacronym{meop}{MEoP}{Manufacturing Edge of Part}
\newacronym{eom}{EoM}{Excess of Material}


\input{./Tex_Files/header_footer.tex}
\newcommand{\clearemptydoublepage}{
    \newpage{\pagestyle{empty}\cleardoublepage}
}

\newcommand\blankpage{%
    \null
    \thispagestyle{empty}%
    \addtocounter{page}{-1}%
    \newpage}



%   ---   COMIENZO DEL DOCUMENTO   ---   %

\begin{document}


%   ---   INCLUIR PORTADA   ---   %

\input{./Tex_Files/portada.tex}



%   ---   INDICE Y LISTAS   ---   %

% Indice
\pagenumbering{gobble}
\tableofcontents
\newpage

% Numeracion en romano
\pagenumbering{roman}
\raggedbottom

% Figuras    
\addcontentsline{toc}{section}{\listfigurename}
\listoffigures
\clearpage

% Tablas
\renewcommand{\listtablename}{Índice de tablas}
\addcontentsline{toc}{section}{\listtablename}
\listoftables
\clearpage

% Acronyms  
\renewcommand{\acronymname}{Acrónimos}
\addcontentsline{toc}{section}{\acronymname}
\printglossary[type=\acronymtype]   % \printglossary[type=\acronymtype,style=long]
\clearpage

% Numeracion en arabico
\setcounter{page}{0}
\pagenumbering{arabic}



%   ---   ARCHIVOS DEL DOCUMENTO   ---   %
% Es recomendable escribir el trabajo en documentos separados y luego importarlos al main.

\chapter{Capítulo} \label{ch:chapter_01}

En este capítulo se muestran varios ejemplos.

\textbf{Negrita} \textit{cursiva}

Hacer una referencia, en este caso al \autoref{ch:chapter_01}.

% ---   ---   --- %


\section{Sección}

\subsection{Citar referencias y acrónimos}

\cite{im78552}, \acrshort{mc}, \acrfull{mc}.

\subsection{Enumeraciones}

Enumeración.

\begin{enumerate}
	\item La impresora debe contar con un sistema de nivelación de la base de impresión.
	\item El sistema de extrusión de filamento debe asegurar que no se producirán inconsistencias durante los periodos largos de trabajo.
\end{enumerate}

Enumeración cambiando los items.

\begin{enumerate}[label= \textbf{R-\arabic*}]
	\item La impresora debe contar con un sistema de nivelación de la base de impresión.
	\item El sistema de extrusión de filamento debe asegurar que no se producirán inconsistencias durante los periodos largos de trabajo. \label{req:extrusion}
\end{enumerate}

Referenciar un item: \ref{req:extrusion}, \autoref{req:extrusion}.


Ejemplo de bulletpoints.


\begin{itemize}[label={\scriptsize\raisebox{0.5ex}{\textbullet}}]

	\item Perfilería de aluminio.

\end{itemize}



%   ---   ---   %

\subsection{Figuras}

Las figuras se pueden fijar en el texto con H, posicionarlas lo mejor posible con h!, arriba con t, etc.

\begin{figure}[H]
	\centering
	\includegraphics[width=100mm]{duet3}
	\caption{Motherboard Duet 3 6HC.}
	\label{fig:000_00}
\end{figure}

Subfiguras en paralelo.

\begin{figure}[H]
	\centering
	\subfloat[Vista frontal del montaje.\label{fig:mont1}]
	{
		\includegraphics[width=50mm,angle=0]{Mont_01}
	}
	\hspace*{10mm}
	\subfloat[Vista trasera del montaje.\label{fig:mont2}]
	{\includegraphics[width=50mm,angle=0]{Mont_02}
	}
	\caption{Montaje del sistema de transmisión del eje X}
	\label{fig:mont_nema}
\end{figure}


Ejemplo dos figuras en paralelo centradas verticalemtne.

\begin{figure}[H]
	\begin{minipage}{\textwidth}
		\centering
		\raisebox{-0.5\height}{\includegraphics[width=0.4\textwidth]{Perfil_Aluminio}}
	\hspace*{.2in}
		\raisebox{-0.5\height}{\includegraphics[width=0.25\textwidth]{Perfil_Aluminio_Union}
		}
	\end{minipage}
	\caption{Ejemplos de perfilería de aluminio.}
	\label{fig:perfileria_alumnio}
\end{figure}

Barrera que no pueden atravesar las figuras.
\FloatBarrier


%   ---   ---   %

\subsection{Ecuaciones}

Ejemplo de ecuación, \autoref{eq:velocidad}

\begin{equation}\label{eq:velocidad}
	l = \frac{\pi\cdot1.8}{180\cdot P}\cdot r = \frac{\pi\cdot1.8}{180\cdot16}\cdot 6 = 0.0117 \qquad [\mathrm{mm}].
\end{equation}



% ---   ---   --- %

\subsection{Tablas}

Ejemplo de tabla de grandes dimensiones e introducir una página apaisada. También se muestra como poner en negrita letras griegas, que a veces dan problemas.

\begin{landscape}
    \vspace*{\fill}
    \input{Tables/table_example}
    \vspace*{\fill}
    \clearpage
\end{landscape}


% ---   ---   --- %

\subsection{Código}


\begin{lstlisting}
M303 H0 S60 ; auto tune heater 0, default PWM (100%), 60C target
M303        ; report the auto-tune status or last resulM303 ; report the auto-tune status or last result
M500        ; save parameters
\end{lstlisting}

\lstinputlisting{Code/function.m}



\clearpage


%   ---   BIBLIOGRAFÍA/REFERENCIAS   ---   %

\phantomsection
\addcontentsline{toc}{section}{Referencias}
\renewcommand{\refname}{Referencias}
\bibliographystyle{elsarticle-num}
\bibliography{references.bib}
\clearpage



%   ---   ANEXOS   ---   %

\appendix
\renewcommand{\appendixname}{Anexos}
\addcontentsline{toc}{section}{\appendixname}
\clearpage % or \cleardoublepage
\appendixpage
\addappheadtotoc
\renewcommand{\appendixname}{Anexo}


\chapter{Título del anexo} \label{ap:montaje}

Aquí puedes meter la información que no sea imprescindible en el cuerpo del trabajo pero si que interese que esté en el documento.
\clearpage


\end{document}